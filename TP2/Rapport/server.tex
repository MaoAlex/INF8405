Les différents utilisateurs doivent communiquer entre eux afin de pouvoir voter, mais aussi pour connaître la position de chacun
des utilisateurs. Une telle communication nécessite la présence d'un serveur connu de tous. Pour mettre en place une serveur nous
avons choisi d'utiliser le service cloud \textit{Firebase}. Les données sont stockées sous le format JSON sur le serveur. On peut
alors les récupérer via leur id (générés par l'API Firebase).
\newline

Cette solution a plusieurs avantages:
\begin{enumerate}
    \item la présence d'une API déjà testée et éprouvée, ce qui limite le nombre de bugs. Et permet de ne pas réinventer la roue.
    \item la présence de mécanismes pour écouter des données sur le serveur et être informé en cas de modifications.
    \item une grande communauté d'utilisateurs, ce qui facilite la recherche d'information en cas de déboguage.
\end{enumerate}

Ainsi, les coordonnées (longitude/latitude) de chaque utilisateur peuvent être mises à jour sur le serveur en temps réel (ou
presque). Il suffit pour cela d'envoyer ces nouvelles coordonnées sur le serveur à chaque fois que le \textit{callback} associée
au changement de position dans \textit{Google Maps} est appelée (l'envoi se ferait dans le dit \textit{callback}). Les autres membres
du groupe n'auraient qu'à écouter ces coordonnées sur le serveur afin d'être informé de tout changement de position.
\newline

Cette approche est un peu gourmande en terme de réseau, aussi il est possible de mettre à jour les données uniquement à des moments jugés
opportuns. On obtient alors une meilleure utilisation du réseau mais on perd le temps réel.
\newline

Nous voulions abstraire au maximum le fonctionnement du serveur ainsi que la récupération des données géographiques. Aussi, nous
encapsulons ces mécanismes dans une classe \textit{ConnectedMap}. Cette classe se charge d'implémenter les \textit{callback}
récupérant les informations de position. Elle se charge également d'instancier la classe \textit{firebaseBD}.
\newline

Cette classe, se charge d'interfacer l'API \textit{Firebase} afin de fournir simplement toutes les fonctionnalités nécessaires pour
ce projet. Elle permet entre autres, de récupérer un utilisateur ou un groupe depuis le serveur. Les réponses de la base de
données étant asynchrones, le comportement attendu lors de la récupération des données doit être encapsulé dans une
\textit{callback} (qui sera appelée après récupération des données). Par exemple, si on veut récupérer un utilisateur, si on a une
classe \textit{User} qui contient tous les attributs nécessaires pour identifier l'utilisateur. On encapsulera sûrement cette
classe dans une autre, disons \textit{LocalUser}. La récupération se fera alors en plusieurs étapes:
\begin{enumerate}
    \item création d'un objet \textit{LocalUser}
    \item création et assignation d'une \textit{callback} à l'objet précédemment créée.
    \item appel d'une méthode du type \textit{getUser(id, localUser)}
    \item lorsque l'objet est récupéré du serveur,  la \textit{callback} est appellée (on peut par exemple afficher une fenêtre
        qui indique que les données ont bien été sauvegardées)
\end{enumerate}
