Les différents utilisateurs doivent communiquer entre eux afin de pouvoir voter, mais aussi pour connaître la position de chacun
des utilisateurs. Une telle communication nécessite la présence d'un serveur connu de tous. Pour mettre en place une serveur nous
avons choisi d'utiliser le service cloud \textit{firebase}. Les données sont stockées sous le format jason sur le serveur. On peut
alors les récupérer via leur id (générés par l'API firebase).
\newline

Cette solution a plusieurs avantages:
\begin{enumerate}
    \item la présence d'une API déjà testée et éprouvée, ce qui limite le nombre de bugs. Et permet de ne pas réinventer la roue.
    \item la présence de mécanismes de pour écouter des données sur le serveur et être informé en cas de modification.
    \item une grande communauté d'utilisateurs, ce qui facilite la recherche d'information en cas de débogguage.
\end{enumerate}

Ainsi, les coordonnées (longitude/latitude) de chaque utilisateur peuvent être mises à jour sur le serveur en temps réel (ou
presque). Il suffit pour cela d'envoyer ces nouvelles coordonnées sur le serveur à chaque fois que la \textit{callback} associée
au changement de position dans \textit{maps} est appelée. (L'envoie se ferait dans la dite \textit{callback}). Les autres membres
du groupe n'aurait qu'à écouter ces coordonnées sur le serveur afin d'être informé de tous changements de position.
\newline

Cette approche est un peu gourmande en terme de réseau, aussi il est possible de mettre à jour les données que à des moments jugés
opportuns. On obtient alors une meilleure utilisation du réseau mais on perd le temps réel.
