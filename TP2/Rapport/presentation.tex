Une première fonctionalité, qui nous semblait importante est la capacité de s'authentifier sur n'importe quel mobile. C'est-à-dire
que l'utilisateur peut posséder  plusieurs appareils mobiles (plusieurs portables ou encore un portable et un tablette), et il
pourra s'authentifier avec le même compte sur tous ces appareils. L'utilisateur peut, sur la page de login, s'identifier va un
motde passe et l'adresse mail associée à son compte. Lors de la création du compte, ses derniers sont stockés sur le serveur.
Au moment du login, on interroge la base de données pour savoir si le mot de passe rentré par l'utilisateur est bien celui qui est
associé à ce compte. Si l'utilisateur utilise un nouvel appareil, ses données seront téléchargées depuis le serveur.
\newline

Il est également possible de créer, sur la page associée au groupe, un nouveau groupe et de le sauvegarder sur le serveur. Ce
serveur contient toutes les données relatives aux utilisateurs est aux groupes. Cela permet potentiellement à chaque utilisateur
de récupérer les préférences de tous les autres et donc d'organiser des rencontres en fonction des préférences de chacun.
\newline

Afin, dans une certaine mesure, de permettre le fonctionnement hors-ligne, une base de données sql lite est également présente.
Elle permet de récupérer les données des autrres utilisateurs afin de potentiellement consulter leur profil ou encore la fiche de
description des groupes. De même, elle permettrait de consulter les événements déjà prévus pour un groupe, ainsi que de consulter
leur description (si celle-ci a été rentrée). Cette fonctionnalité permettrait de mieux se préparer pour les événements à venir.
