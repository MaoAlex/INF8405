L'objectif du TP n°2 était de développer une application destinée à faciliter les rencontres entre membres d'un groupe. Plus précisément, les requis de l'application étaient centrés autour de l'utilisation de différentes ressources, comme le GPS, le calendrier des membres, des API externes comme Google Maps ou Google Places. L'application devait permettre à chaque personne du groupe de définir son profil (nom, courriel, lieux de rencontre préférés) et s'il était organisateur ou non. Une fois ceci fait, et le groupe constitué, l'application, en se basant sur les lieux de rencontre préférés et la position géographique des membres du groupe, proposait trois options de rendez-vous. Les membres du groupe devaient alors choisir chacun l'option qui leur convenait le mieux. Le résultat était transmis à l'organisateur qui pouvait alors définir un lieu et une heure, d'après les disponibilités des membres du groupes, ainsi qu'un petit texte présentant le choix qu'il a retenu. Les informations étaient alors transmises à tous les membres du groupe afin que ceux-ci disposent des informations nécessaires pour aller au rendez-vous.\\
D'un point de vue technique, l'application a pour requis d'être exécutable sur les plateformes Android supportant l'API 18 et supérieures (jusqu'à la version 23 pour la plus récente), c'est à dire des versions d'Android 4.3 à Android 6.0. L'application doit être capable d'être utilisable, avec des fonctionnalités réduites, même lorsque la connexion au réseau cellulaire ou Wi-Fi n'est pas possible, ce qui a eu un impact sur les choix d'implémentation réalisés lors du développement de cette application.  
