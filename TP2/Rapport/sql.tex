Si la partie serveur utilise Firebase pour stocker les différentes données, il a fallu réfléchir à un moyen alternatif de disposer des dernières données à jour en local, sur l'appareil mobile de l'utilisateur, au cas où celui ci ne disposerait pas d'une connexion Internet à un moment donné de son utilisation de l'application. Nous avons opté pour une solution commune à ce problème sous Android, à savoir l'existence d'une base de données SQLite sur l'appareil. \\
On peut notablement remarquer que la majorité des objets encapsulant les données de notre application sont doublés d'une autre classe gérant la table associée à ces objets dans la base de données SQLite utilisée. Ces classes ont pour objectifs, entre autres, la définition des tables, la définition des requêtes de récupération, d'insertion et de mise-à-jour des données contenues, ainsi que de créer les objets associés pour les récupérer sous une forme plus adaptée à nos besoins.\\
Lors de la conception de ces bases de données, il est important de noter quelques points pris particulièrement en compte :
\begin{itemize}
\item La création de colonnes numériques destinées à être des identifiants pour lier les éléments de différentes tables et
    permettre plus facilement de rattacher les éléments allant ensemble dans chacune des tables. On peut noter par exemple que la
    table Groupe réfère les utilisateurs en faisant partie par leur user\_id, ce qui permet de réaliser des jointures et récupérer en une seule requête les informations sur les différents membres d'un même groupe.
\item Cette base de données doit nous permettre de représenter fidèlement le dernier état connu de la base de données en ligne. Elle est donc mise à jour régulièrement et permet de réaliser de manière assez transparente les traitements que nous voulons effectuer à un moment donnée, que la base de données en ligne soit accessible ou non.
\end{itemize}
