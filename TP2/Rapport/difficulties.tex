Comme tous les projets intéressants, celui-ci n'a pas été sans écueils. Ce projet comporte de nombreuses facettes, il doit
posséder une base de données internes, un serveur et pouvoir communiquer avec les API google afin de récupérer des données sur la
position de l'utilisateur courant et sur les lieux (cafés, restaurants, bars) proches d'une certaine position. Il y a donc des
domaines différents à explorer. Le plus simple est alors de répartir ces domaines entre les membres du groupe. Mais s'il chacun
travaille trop de son côté, il devient compliquer d'intégrer les différents travaux respectifs. Le problème est donc avant tout un
problème de communication.
\newline

Cette difficulté est d'autant plus accentuée que ces facettes ne sont pas indépendantes: la base de données doit connaître les
positions afin d'informer les autres utilisateurs lorsqu'ils en font la demande. De ce fait, Les différentes facettes seront très
liées. Et leurs codes respectifs vont, à terme, s'entrecroiser. Ce qui rend leur intégration beaucoup plus compliquée.

\newline
Un autre problème rencontré a été beaucoup plus contraignant, car il a impacté fortement notre travail et nous en a même fait perdre une partie non négligeable (environ une journée de travail). Il s'agit du dépôt Git utilisé, qui a rencontré un problème majeur nous ayant fait perdre les derniers commits réalisés au cours d'une journée relativement chargée. Alexandre a notamment été fortement handicapé par ce problème car il disposait des dernières versions testées des classes traitant de la base de données. Le retard accumulé dans ce domaine a fini par se répercuter sur le développement d'autres fonctionnalités qui se retrouvaient en suspens, car dépendantes de la présence d'une base de données locales fonctionnelle. Nous avons essayé au maximum de limiter les dommages occasionnés sur notre travail, mais la marge de temps était assez restreinte. Certaines fonctionnalités des requis en ont probablement pâti. 

Enfin, nous avons rencontré des difficultés quand au choix de la solution pour le serveur. \textit{Google Drive} était par exemple proposé ou en tout cas mentionné dans le sujet, mais ce service ne nous paraissait pas approprié pour un tel usage. L'API Android ne permet tout simplement pas de récupérer des fichiers qui
ont été créés par le programme faisant appel à l'API. De plus, le partage des fichiers entre différents utilisateurs est
laborieux. Et l'utiliser comme serveur demanderait  de réimplémenter de nombreux mécanismes. en particulier il faudrait
réimplémenter des mécanismes d'écoute et de mise à jour. Nous avons donc eu du mal ensuite à nous décider parmi les nombreuses alternatives gratuites disponibles en ligne pour effectuer ce genre de tâche. Nous avons finalement opté pour Firebase, mais nous avons perdu une certaine quantité de temps qui nous aurait été utile dans la dernière ligne droite de ce TP.