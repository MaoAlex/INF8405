L'API Google Places se décompose en plusieurs composantes distinctes. Il existe notamment une version de l'API pour Android et une version Web Services destinée à être utilisée du côté serveur d'une application. Celle-ci étant la seule API proposant la recherche de Places dans un certain rayon autour d'un point, c'est celle que nous avons naturellement privilégiée, même si son utilisation telle que nous l'avons faite n'était pas recommandée par Google. En effet, dans le cadre d'un projet plus conséquent, nous aurions dû en réalité manipuler cette API par l'intermédiaire d'un serveur que nous aurions mis en place, plutôt qu'en faisant directement la requête à partir de l'application. Cependant, dans un soucis de simplification et puisque l'application n'a pas pour vocation d'être publiée, nous avons choisi d'effectuer directement la requête depuis notre application, ce qui a quelques effets sur la sécurité (la clé API Google Places de l'application peut être volée en interceptant les URLs appelées par l'application).\\
Nous avons donc utilisé l'API Google Places Web Services qui présente les caractéristiques nécessaires à la réalisation de l'application en suivant les requis. Pour ce faire, nous avons procédé en différentes étapes :
\begin{itemize}
\item Nous avons tout d'abord créé un objet SearchCriteria chargé d'encapsuler les paramètres de la requête à effectuer : ces paramètres sont de diverses sortes : le point géographique autour duquel nous effectuerons la recherche, le rayon (en mètres) autour du centre dans lequel nous souhaitons avoir les résultats, et enfin le type d'établissements recherché (restaurant, café, etc).
\item Nous avons ensuite créé une tâche asynchrone afin que l'application ne soit pas bloquée pendant le téléchargement du fichier JSON (réponse de l'API). Ceci est d'autant plus crucial lorsqu'on traite d'applications mobiles, qui sont plus à même de fonctionner sur un réseau peu stable et lent.
\item Cette tâche asynchrone construit l'URL via un URIBuilder en fonction du SearchCriteria passé en paramètre. Elle appelle ensuite cette URL et récupère le stream réponse de l'API.
\item Lorsque le téléchargement du stream réponse est terminé, la méthode onPostExecute est automatiquement appelée par Android. On s'en sert alors pour transmettre les données récupérées à une méthode qui parsera le JSON obtenu.
\end{itemize} 

A l'issue de ces étapes, notre utilisation de l'API Google Places est terminée. Il nous reste alors à utiliser la réponse obtenue pour enrichir notre application. Nous avons choisi d'encapsuler les éléments caractéristiques d'un lieu dans un objet particulier. La structure des résultats étant connue et présentée dans la documentation de l'API Google Places, le traitement suivant peut être mis en place :
\begin{itemize}
\item Les résultats sont compris dans un tableau JSON étiqueté par la clé "results". On récupère alors un JSONArray dont chaque entrée est une Place retournée par Google Places.
\item Pour chacun des éléments de ce JSONArray, nous récupérons la valeur associée à l'étiquette "name", le nom de l'établissement, ainsi que ces coordonnées. Ce point est plus problématique car les coordonnées sont imbriquées : il faut d'abord récupérer le contenu de la clé "geometry", puis celle de "position" à l'intérieur de "geometry". La latitude ("lat") et la longitude ("lng") sont ainsi récupérables.
\item Le type d'établissement est obtenu différemment : en effet, l'API Google Places retourne l'ensemble des types associés à l'établissement. Il n'est donc pas possible de savoir lequel des types correspond à ce que nous avons demandé. Pour obtenir un renseignement plus précis, nous passons en paramètre à la fonction de parsage le type associé à la recherche, ce qui permet de remplir l'objet Place avec le type.
\end{itemize}