\documentclass[12pt, a4paper]{article}
\usepackage[utf8]{inputenc}
\usepackage[T1]{fontenc}
\usepackage[a4paper,left=2cm,right=2cm,top=2cm,bottom=2cm]{geometry}
\usepackage[frenchb]{babel}
\usepackage{libertine}
\usepackage{hyperref}
\usepackage{amsfonts}
\usepackage{amssymb}
\usepackage[dvipsnames]{xcolor}
\usepackage[pdftex]{graphicx}

\setlength{\parindent}{0cm}
\setlength{\parskip}{1ex plus 0.5ex minus 0.2ex}
\newcommand{\hsp}{\hspace{20pt}}
\newcommand{\HRule}{\rule{\linewidth}{0.5mm}}

\begin{document}

\begin{titlepage}
  \begin{sffamily}
  \begin{center}

    % Upper part of the page. The '~' is needed because \\
    % only works if a paragraph has started.
    \includegraphics[scale=1]{Images/polytechnique_genie_gauche_fr_rgb.png}~\\[1.5cm]

    \textsc{\LARGE École Polytechnique Montréal}\\[2cm]

    \textsc{\Large INF8405 : Informatique Mobile}\\[1.5cm]

    % Title
    \HRule \\[0.4cm]
    { \huge \bfseries TP1 : FlowFree\\[0.4cm] }

    \HRule \\[2cm]
    \includegraphics[scale=0.5]{Images/polytechnique_genie_gauche_fr_rgb.png}
    \\[2cm]

    % Author and supervisor
    \begin{minipage}{0.4\textwidth}
      \begin{flushleft} \large
          Philippe TROCLET \textsc{1815208}\\
          Alexandre  MAO \textsc{1813566}\\
          Fabien  BERQUEZ \textsc{1800325}\\
      \end{flushleft}
    \end{minipage}
    \begin{minipage}{0.4\textwidth}
      \begin{flushright} \large
        \emph{Soumis à :} M. Aurel Josas RANDOLF\\
        \emph{Soumis le :} 17 Février 2016 
      \end{flushright}
    \end{minipage}

    \vfill

  \end{center}
  \end{sffamily}
\end{titlepage}


\section{Introduction}
\section{Présentation Technique}
    \subsection{Fonctionnement de la Grille}
    La grille de jeu est implémentée dans la classe \textit{FlowFreeSimpleGridView}, cette classe étant la classe \textit{View}
afin de posséder toutes les fonctionnalités d'une vue, pour pouvoir intercepter les mouvements de l'utilisateur sur la surface.
Cette vue possèdera un tableau à deux dimensions (nommé grid) représentant les différentes cases de la grille. On distingue deux
types de cases, les cases classiques et les cases contenant un délimiteur. Les cases contenant un délimiteur dans la mesure où,
elles possèdent un cercle de couleur et surtout leur couleur, à tout instant, est toujours du rond présent sur la case. Nous avons
en effet considéré que qu'afin de pouvoir compléter le jeu, que les ronds soit accessibles à tout instant. En effet, dans le cas
contraire, on pourrait se retrouver bloqué dans l'état inachevé car il nous serait impossible de tracer le dernier tube, et donc
d'atteindre l'état de défaite. De ce fait nous voulions pouvoir stoquer deux classes distinctes au sein d'une même structure. De
plus, ces classes on un fonctionnement très proche: elles doivent pouvoir stocker des éléments de tube de couleurs différentes
tout en conservant l'ordre d'arrivée de ces derniers afin de présenter un un affichage cohérent ou le dernier morceau de tube
arrivé sur la case serait celui dessiné en dernier (afin que graphiquement il apparaisse au-dessus de l'autre).
\section{Difficultés rencontrées}
\section{Critiques et suggestions}
\end{document}
